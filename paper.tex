% This template has been tested with LLNCS DOCUMENT CLASS -- version 2.20 (10-Mar-2018)

% !TeX spellcheck = en-US
% !TeX encoding = utf8
% !TeX program = pdflatex
% !BIB program = bibtex
% -*- coding:utf-8 mod:LaTeX -*-

% "a4paper" enables:
%  - easy print out on DIN A4 paper size
%
% One can configure a4 vs. letter in the LaTeX installation. So it is configuration dependend, what the paper size will be.
% This option  present, because the current word template offered by Springer is DIN A4.
% We accept that DIN A4 cause WTFs at persons not used to A4 in USA.

% "runningheads" enables:
%  - page number on page 2 onwards
%  - title/authors on even/odd pages
% This is good for other readers to enable proper archiving among other papers and pointing to
% content. Even if the title page states the title, when printed and stored in a folder, when
% blindly opening the folder, one could hit not the title page, but an arbitrary page. Therefore,
% it is good to have title printed on the pages, too.
%
% It is enabled by default as the springer template as of 2018/03/10 uses this as default

% German documents: pass ngerman as class option
% \documentclass[ngerman,runningheads,a4paper]{llncs}[2018/03/10]
% English documents: pass english as class option
\documentclass[english,runningheads,a4paper]{llncs}[2018/03/10]

%% If you need packages for other papers,
%% START COPYING HERE

% Set English as language and allow to write hyphenated"=words
%
% In case you write German, switch the parameters, so that the command becomes
%\usepackage[english,main=ngerman]{babel}
%
% Even though `american`, `english` and `USenglish` are synonyms for babel package (according to https://tex.stackexchange.com/questions/12775/babel-english-american-usenglish), the llncs document class is prepared to avoid the overriding of certain names (such as "Abstract." -> "Abstract" or "Fig." -> "Figure") when using `english`, but not when using the other 2.
% english has to go last to set it as default language
\usepackage[ngerman,main=english]{babel}
%
% Hint by http://tex.stackexchange.com/a/321066/9075 -> enable "= as dashes
\addto\extrasenglish{\languageshorthands{ngerman}\useshorthands{"}}
%
% Fix by https://tex.stackexchange.com/a/441701/9075
\usepackage{regexpatch}
\makeatletter
\edef\switcht@albion{%
  \relax\unexpanded\expandafter{\switcht@albion}%
}
\xpatchcmd*{\switcht@albion}{ \def}{\def}{}{}
\xpatchcmd{\switcht@albion}{\relax}{}{}{}
\edef\switcht@deutsch{%
  \relax\unexpanded\expandafter{\switcht@deutsch}%
}
\xpatchcmd*{\switcht@deutsch}{ \def}{\def}{}{}
\xpatchcmd{\switcht@deutsch}{\relax}{}{}{}
\edef\switcht@francais{%
  \relax\unexpanded\expandafter{\switcht@francais}%
}
\xpatchcmd*{\switcht@francais}{ \def}{\def}{}{}
\xpatchcmd{\switcht@francais}{\relax}{}{}{}
\makeatother

\usepackage{ifluatex}
\ifluatex
  \usepackage{fontspec}
  \usepackage[english]{selnolig}
\fi

\iftrue % use default-font
  \ifluatex
    % use the better (sharper, ...) Latin Modern variant of Computer Modern
    \setmainfont{Latin Modern Roman}
    \setsansfont{Latin Modern Sans}
    \setmonofont{Latin Modern Mono} % "variable=false"
    %\setmonofont{Latin Modern Mono Prop} % "variable=true"
  \else
    % better font, similar to the default springer font
    % cfr-lm is preferred over lmodern. Reasoning at http://tex.stackexchange.com/a/247543/9075
    \usepackage[%
      rm={oldstyle=false,proportional=true},%
      sf={oldstyle=false,proportional=true},%
      tt={oldstyle=false,proportional=true,variable=false},%
      qt=false%
    ]{cfr-lm}
  \fi
\else
  % In case more space is needed, it is accepted to use Times New Roman
  \ifluatex
    \setmainfont{TeX Gyre Termes}
    \setsansfont[Scale=.9]{TeX Gyre Heros}
    % newtxtt looks good with times, but no equivalent for lualatex found,
    % therefore tried to replace with inconsolata.
    % However, inconsolata does not look good in the context of LNCS ...
    %\setmonofont[StylisticSet={1,3},Scale=.9]{inconsolata}
    % ... thus, we use the good old Latin Modern Mono font for source code.
    \setmonofont{Latin Modern Mono} % "variable=false"
    %\setmonofont{Latin Modern Mono Prop} % "variable=true"
  \else
    % overwrite cmodern with the Times variant
    \usepackage{newtxtext}
    \usepackage{newtxmath}
    \usepackage[zerostyle=b,scaled=.9]{newtxtt}
  \fi
\fi

\ifluatex
\else
  % fontenc and inputenc are not required when using lualatex
  \usepackage[T1]{fontenc}
  \usepackage[utf8]{inputenc} %support umlauts in the input
\fi

\usepackage{graphicx}

% backticks (`) are rendered as such in verbatim environment. See https://tex.stackexchange.com/a/341057/9075 for details.
\usepackage{upquote}

% Nicer tables (\toprule, \midrule, \bottomrule - see example)
\usepackage{booktabs}

%extended enumerate, such as \begin{compactenum}
\usepackage{paralist}

%put figures inside a text
%\usepackage{picins}
%use
%\piccaptioninside
%\piccaption{...}
%\parpic[r]{\includegraphics ...}
%Text...

% For easy quotations: \enquote{text}
% This package is very smart when nesting is applied, otherwise textcmds (see below) provides a shorter command
\usepackage{csquotes}

% For even easier quotations: \qq{text}
\usepackage{textcmds}

%enable margin kerning
\RequirePackage[%
  babel,%
  final,%
  expansion=alltext,%
  protrusion=alltext-nott]{microtype}%
% \texttt{test -- test} keeps the "--" as "--" (and does not convert it to an en dash)
\DisableLigatures{encoding = T1, family = tt* }

%tweak \url{...}
\usepackage{url}
%\urlstyle{same}
%improve wrapping of URLs - hint by http://tex.stackexchange.com/a/10419/9075
\makeatletter
\g@addto@macro{\UrlBreaks}{\UrlOrds}
\makeatother
%nicer // - solution by http://tex.stackexchange.com/a/98470/9075
%DO NOT ACTIVATE -> prevents line breaks
%\makeatletter
%\def\Url@twoslashes{\mathchar`\/\@ifnextchar/{\kern-.2em}{}}
%\g@addto@macro\UrlSpecials{\do\/{\Url@twoslashes}}
%\makeatother

% Diagonal lines in a table - http://tex.stackexchange.com/questions/17745/diagonal-lines-in-table-cell
% Slashbox is not available in texlive (due to licensing) and also gives bad results. This, we use diagbox
%\usepackage{diagbox}

% Required for package pdfcomment later
\usepackage{xcolor}

% For listings
\usepackage{listings}
\lstset{%
  basicstyle=\ttfamily,%
  columns=fixed,%
  basewidth=.5em,%
  xleftmargin=0.5cm,%
  captionpos=b}%
\renewcommand{\lstlistingname}{List.}
% Fix counter as described at https://tex.stackexchange.com/a/28334/9075
\usepackage{chngcntr}
\AtBeginDocument{\counterwithout{lstlisting}{section}}

% Enable nice comments
\usepackage{pdfcomment}
%
\newcommand{\commentontext}[2]{\colorbox{yellow!60}{#1}\pdfcomment[color={0.234 0.867 0.211},hoffset=-6pt,voffset=10pt,opacity=0.5]{#2}}
\newcommand{\commentatside}[1]{\pdfcomment[color={0.045 0.278 0.643},icon=Note]{#1}}
%
% Compatibality with packages todo, easy-todo, todonotes
\newcommand{\todo}[1]{\commentatside{#1}}
% Compatiblity with package fixmetodonotes
\newcommand{\TODO}[1]{\commentatside{#1}}

% Bibliopgraphy enhancements
%  - enable \cite[prenote][]{ref}
%  - enable \cite{ref1,ref2}
% Alternative: \usepackage{cite}, which enables \cite{ref1, ref2} only (otherwise: Error message: "White space in argument")

% Doc: http://texdoc.net/natbib
\usepackage[%
  square,        % for square brackets
  comma,         % use commas as separators
  numbers,       % for numerical citations;
%  sort,          % orders multiple citations into the sequence in which they appear in the list of references;
  sort&compress, % as sort but in addition multiple numerical citations
                 % are compressed if possible (as 3-6, 15);
]{natbib}
% In the bibliography, references have to be formatted as 1., 2., ... not [1], [2], ...
\renewcommand{\bibnumfmt}[1]{#1.}

\ifluatex
  % does not work when using luatex
  % see: https://tex.stackexchange.com/q/419288/9075
\else
  % Prepare more space-saving rendering of the bibliography
  % Source: https://tex.stackexchange.com/a/280936/9075
  \SetExpansion
  [ context = sloppy,
    stretch = 30,
    shrink = 60,
    step = 5 ]
  { encoding = {OT1,T1,TS1} }
  { }
\fi

% Put footnotes below floats
% Source: https://tex.stackexchange.com/a/32993/9075
\usepackage{stfloats}
\fnbelowfloat

% Enable that parameters of \cref{}, \ref{}, \cite{}, ... are linked so that a reader can click on the number an jump to the target in the document
\usepackage{hyperref}
% Enable hyperref without colors and without bookmarks
\hypersetup{hidelinks,
  colorlinks=true,
  allcolors=black,
  pdfstartview=Fit,
  breaklinks=true}
%
% Enable correct jumping to figures when referencing
\usepackage[all]{hypcap}

\usepackage[group-four-digits,per-mode=fraction]{siunitx}

%enable \cref{...} and \Cref{...} instead of \ref: Type of reference included in the link
\usepackage[capitalise,nameinlink]{cleveref}
%Nice formats for \cref
\usepackage{iflang}
\IfLanguageName{ngerman}{
  \crefname{table}{Tab.}{Tab.}
  \Crefname{table}{Tabelle}{Tabellen}
  \crefname{figure}{\figurename}{\figurename}
  \Crefname{figure}{Abbildung}{Abbildungen}
  \crefname{equation}{Gleichung}{Gleichungen}
  \Crefname{equation}{Gleichung}{Gleichungen}
  \crefname{listing}{\lstlistingname}{\lstlistingname}
  \Crefname{listing}{Listing}{Listings}
  \crefname{section}{Abschnitt}{Abschnitte}
  \Crefname{section}{Abschnitt}{Abschnitte}
  \crefname{paragraph}{Abschnitt}{Abschnitte}
  \Crefname{paragraph}{Abschnitt}{Abschnitte}
  \crefname{subparagraph}{Abschnitt}{Abschnitte}
  \Crefname{subparagraph}{Abschnitt}{Abschnitte}
}{
  \crefname{section}{Sect.}{Sect.}
  \Crefname{section}{Section}{Sections}
  \crefname{listing}{\lstlistingname}{\lstlistingname}
  \Crefname{listing}{Listing}{Listings}
}


%Intermediate solution for hyperlinked refs. See https://tex.stackexchange.com/q/132420/9075 for more information.
\newcommand{\Vlabel}[1]{\label[line]{#1}\hypertarget{#1}{}}
\newcommand{\lref}[1]{\hyperlink{#1}{\FancyVerbLineautorefname~\ref*{#1}}}

\usepackage{xspace}
%\newcommand{\eg}{e.\,g.\xspace}
%\newcommand{\ie}{i.\,e.\xspace}
\newcommand{\eg}{e.\,g.,\ }
\newcommand{\ie}{i.\,e.,\ }

%introduce \powerset - hint by http://matheplanet.com/matheplanet/nuke/html/viewtopic.php?topic=136492&post_id=997377
\DeclareFontFamily{U}{MnSymbolC}{}
\DeclareSymbolFont{MnSyC}{U}{MnSymbolC}{m}{n}
\DeclareFontShape{U}{MnSymbolC}{m}{n}{
  <-6>    MnSymbolC5
  <6-7>   MnSymbolC6
  <7-8>   MnSymbolC7
  <8-9>   MnSymbolC8
  <9-10>  MnSymbolC9
  <10-12> MnSymbolC10
  <12->   MnSymbolC12%
}{}
\DeclareMathSymbol{\powerset}{\mathord}{MnSyC}{180}

\ifluatex
\else
  % Enable copy and paste - also of numbers
  % This has to be done instead of \usepackage{cmap}, because it does not work together with cfr-lm.
  % See: https://tex.stackexchange.com/a/430599/9075
  \input glyphtounicode
  \pdfgentounicode=1
\fi

% correct bad hyphenation here
\hyphenation{op-tical net-works semi-conduc-tor}

%% END COPYING HERE


% Add copyright
% Do that for the final version or if you send it to colleagues
\iffalse
  %state: intended|submitted|llncs
  %you can add "crop" if the paper should be cropped to the format Springer is publishing
  \usepackage[intended]{llncsconf}

  \conference{name of the conference}

  %in case of "llncs" (final version!)
  %example: llncs{Anonymous et al. (eds). \emph{Proceedings of the International Conference on \LaTeX-Hacks}, LNCS~42. Some Publisher, 2016.}{0042}
  \llncs{book editors and title}{0042} %% 0042 is the start page
\fi

% For demonstration purposes only
\usepackage[math]{blindtext}
\usepackage{mwe}
% ----------------------------------------------------------------------------------------------------------------------
% ----------------------------------------------------------------------------------------------------------------------
\begin{document}

\title{Zukünftige Entwicklungen in VANETs}
%If Title is too long, use \titlerunning
%\titlerunning{Short Title}

%Single insitute
\author{Fabian Sieber}
%If there are too many authors, use \authorrunning
%\authorrunning{First Author et al.}
\institute{Technische Universität Dortmund}

%% Multiple insitutes - ALTERNATIVE to the above
% \author{%
%     Firstname Lastname\inst{1} \and
%     Firstname Lastname\inst{2}
% }
%
%If there are too many authors, use \authorrunning
%  \authorrunning{First Author et al.}
%
%  \institute{
%      Insitute 1\\
%      \email{...}\and
%      Insitute 2\\
%      \email{...}
%}

\maketitle

\begin{abstract}
  Der Fortschritt in der Kommunikationstechnik, wie die Einführung von 5G, hat in den letzten Jahren zu einem wachsenden Interesse an Kommunikation mit und zwischen Fahrzeugen geführt.
  Dies beinhaltet sowohl Anwendungen im Bereich der Unterhaltung, als auch im Bereich von Assistenzsystemen und der Verkehrsregelung.
  Aufgrund dieser Zunahme von Kapazitäten innerhalb eines Fahrzeugs bietet sich eine zusätzliche Verwendung der verbauten Technik an.
  Dazu wird in dieser Ausarbeitung das Konzept von \textit{Vehicular Clouds} vorgestellt und zum herkömmlichen \textit{Cloud Computing} abgegrenzt.
  Außerdem werden mögliche Anwendungen vorgestellt, welche die durch die Nutzung von \textit{Vehicular Clouds} neuentstehenden Chancen beleuchten.
  Sicherheit und Privatsphäre sind kritische Aspekte, auf die im späteren Verlauf dieser Ausarbeitung eingegangen wird.
  Zusätzlich werden mögliche Architekturen und die Aggregation von Ressourcen betrachtet\cite{conti2013mobile}.
\end{abstract}

\begin{keywords}
  Mobile ad-hoc Netze, VANET, Cloud Computing, Vehicular Cloud
\end{keywords}
% ----------------------------------------------------------------------------------------------------------------------
\section{Einleitung}\label{sec:intro}
  Der Fortschritt in der Kommunikationstechnik, wie die Einführung von 5G, hat in den letzten Jahren zu einem wachsenden Interesse an Kommunikation mit und zwischen Fahrzeugen geführt.
  Anfängliche Vorstellungen der zukünftigen Entwicklungen versprechen eine verstärkte Kommunikation zwischen den Fahrzeugen, welche durch den Austausch von Informationen über die Strassenverhältnisse oder andere verkehrsbezogene Gefahren zu einem sichereren Verkehr auf unseren Autobahnen und Straßen führt.
  Diese Aussicht führte zu einer vermehrten Ausstattung von Fahrzeugen mit Kommuniationstechnik und der Einführung von Verkehrstelematik (\textit{Intelligent Transportations Systems ITS})\cite{conti2013mobile}.
  Ein Beispiel dafür sind an Autobahnen montierte Wechselverkehrszeichen, welche über Starßenverhältnisse informieren, flexible Geschwindigkeitsbegrenzungen ermöglichen oder auch die Öffnung und Schließung von Fahrspuren ankündigen.
  Mithilfe dieser können beispielsweise bei einem Unfall die betroffenen Spuren gesperrt werden, um die Unfallstelle besonders bei nicht vorhandener Geschwindigkeitsbegrenzung frühzeitig zu sichern, oder auch in Hauptverkehrszeiten der Standstreifen als zusätzliche Fahrspur freigegeben werden.

  In den Vereinigten Staaten von Amerika wurde durch die \textit{US Federal Communications Commission (FCC)} ein 75 MHz breites Spektrum im Bereich des 5.850- bis 5.925-GHz-Band für die Nutzung durch \textit{Dedicated Short-Range Communications (DSRC)} reserviert.
  Dies übertrifft die für die verkehrssichernde Kommunikation notwendigen Kapazitäten.
  Deshalb wird erwartet, dass Drittanbieter diese überflüssige Bandbreite nutzen, um eigene Anwendungen zu schaffen, welche den Verkehrsteilnehmern angeboten werden.
  Umgesetzt wurde dies bereits jetzt in Ortsbezogenen Anwendungen oder auch Unterhaltungsmedien.
  Ein Beispiel dafür ist die Integration von Spotify, YouTube und Netflix oder auch Videospielen in Fahrzeugen des Herstellers Tesla.
  Mit der Verbreitung dieser Anwendungen wird auch von dem Ausbau der Infrastruktur entlang den Verkehrswege durch Drittanbieter und die Ausstattung von Fahrzeugen mit fähigerer Technik ausgegangen\cite{conti2013mobile}.

  Der aktuelle Stand der Technik in Fahrzeugen reagiert lediglich auf Ereignisse, welche durch die eigene Sensorik wahrgenommen werden und kann nur auf das eigene Fahrzeug einwirken.
  Ziel ist es, dass diese Informationen und Ressourcen zwischen den Fahrzeugen geteilt werden, um gemeinsam ein Problem zu lösen, welches ohne Kooperation übermäßig lang zu lösen dauern würde (beispielsweise ein Stau) oder mit den begrenzten Ressourcen eines Fahrzeugs gar nicht lösbar wäre.
  Diese Ausarbeitung stellt das Konzept der \textit{Vehicular Clouds} vor, welches die zwischen den Fahrzeugen bereits vorhandenen Netze ``in die Cloud bringt''\cite{conti2013mobile}.
  Dazu wird zunächst in Kapitel \ref{sec:basics} auf einige Grundlagen eingegangen, welche das Konzept und nötiges Wissen vorstellen.
  Daraufhin wird in Kapitel \ref{sec:ccvsvc} die \textit{Vehicular Cloud} vom herkömmlichen \textit{Cloud Computing} abgegrenzt, um die Notwendigkeit von Neuerungen und die entstehenden Herausforderungen zu verdeutlichen.
  In Kapitel \ref{sec:examples} werden zur Darstellung der durch \textit{Vehicular Clouds} entstehenden Möglichkeiten Anwendungsbeispiele vorgestellt.
  Aufgrund der Verbindung der Systeme innerhalb eines Fahrzeugs und der Masse an Daten sind Sicherheit und Privatsphäre besonders wichtig, dies wird in Kapitel \ref{sec:securityprivacy} behandelt.
  Im Anschluss wird in Kapitel \ref{sec:architectures} auf Architekturen eingegangen, welche zur Umsetzung von \textit{Vehicular Clouds} eingesetzt werden können.
  Kapitel \ref{sec:aggregation} befasst sich mit der Art und Weise, wie die Ressourcen in \textit{Vehicular Clouds} verteilt und erlangt werden können.
  Schließlich wird in Kapitel \ref{sec:conclusion} ein Fazit gezogen und ein Ausblick auf weitere Entwicklungen gegeben.
% ----------------------------------------------------------------------------------------------------------------------
\section{Grundlagen}
\label{sec:basics}
In diesem Kapitel wird zunächst auf die bisher verbaute Technik in Fahrzeugen eingegangen und welche Entwicklungen dabei in naher Zukunft absehbar sind.
Des Weiteren werden Fahrzeug-Ad-hoc-Netze vorgestellt, auf denen \textit{Vehicular Clouds} grundlegend basieren.
Außerdem wird auf \textit{Cloud Computing} eingegangen, da darin ebenfalls eine Grundlage besteht und dies im weiteren Verlauf zu \textit{Vehicular Clouds} abgegrenzt werden soll.
Schließlich wird das Konzept der \textit{Vehicular Clouds} vorgestellt.

\subsection{Verbaute Technik}

\subsection{Fahrzeug-Ad-hoc-Netze}

\subsection{Cloud Computing}

\subsection{Vehicular Clouds}
% ----------------------------------------------------------------------------------------------------------------------
\section{Cloud Computing und Vehicular Clouds: Worin liegt der Unterschied?}
\label{sec:ccvsvc}
% ----------------------------------------------------------------------------------------------------------------------
\section{Anwendungsbeispiele}
\label{sec:examples}

\subsection{Beispiel 1}

\subsection{Beispiel 2}

\subsection{Beispiel 3}
% ----------------------------------------------------------------------------------------------------------------------
\section{Sicherheit und Privatsphäre}
\label{sec:securityprivacy}

% ----------------------------------------------------------------------------------------------------------------------
\section{Schlüsselverwaltung}
\label{sec:keymanagement}

% ----------------------------------------------------------------------------------------------------------------------
\section{Mögliche Architekturen}
\label{sec:architectures}

\subsection{Statisch}

\subsection{Dynamisch}
% ----------------------------------------------------------------------------------------------------------------------
\section{Ansätze für Aggregation von Ressourcen}
\label{sec:aggregation}

\subsection{Virtualisierung}

\subsection{Lastverteilung}

% ----------------------------------------------------------------------------------------------------------------------
\section{Fazit und Ausblick}
\label{sec:conclusion}

% ----------------------------------------------------------------------------------------------------------------------
% \subsubsection*{Acknowledgments}
% \ldots

% In the bibliography, use \texttt{\textbackslash textsuperscript} for \qq{st}, \qq{nd}, \ldots:
% E.g., \qq{The 2\textsuperscript{nd} conference on examples}.
% When you use \href{https://www.jabref.org}{JabRef}, you can use the clean up command to achieve that.
% See \url{https://help.jabref.org/en/CleanupEntries} for an overview of the cleanup functionality.

\renewcommand{\bibsection}{\section*{References}} % requried for natbib to have "References" printed and as section*, not chapter*
% Use natbib compatbile splncsnat style.
% It does provide all features of splncs03, but is developed in a clean way.
% Source: http://phaseportrait.blogspot.de/2011/02/natbib-compatible-bibtex-style-bst-file.html
\bibliographystyle{splncsnat}
\begingroup
  \ifluatex
    %try to activate if bibliography looks ugly
    %\sloppy
  \else
    \microtypecontext{expansion=sloppy}
  \fi
  \small % ensure correct font size for the bibliography
  \bibliography{paper}
\endgroup

% Enfore empty line after bibliography
\ \\
%
Alle Links wurden zuletzt am 14. Januar 2021 aufgerufen.
\end{document}
