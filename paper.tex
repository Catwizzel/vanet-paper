% This template has been tested with LLNCS DOCUMENT CLASS -- version 2.20 (10-Mar-2018)

% !TeX spellcheck = en-US
% !TeX encoding = utf8
% !TeX program = pdflatex
% !BIB program = bibtex
% -*- coding:utf-8 mod:LaTeX -*-

% "a4paper" enables:
%  - easy print out on DIN A4 paper size
%
% One can configure a4 vs. letter in the LaTeX installation. So it is configuration dependend, what the paper size will be.
% This option  present, because the current word template offered by Springer is DIN A4.
% We accept that DIN A4 cause WTFs at persons not used to A4 in USA.

% "runningheads" enables:
%  - page number on page 2 onwards
%  - title/authors on even/odd pages
% This is good for other readers to enable proper archiving among other papers and pointing to
% content. Even if the title page states the title, when printed and stored in a folder, when
% blindly opening the folder, one could hit not the title page, but an arbitrary page. Therefore,
% it is good to have title printed on the pages, too.
%
% It is enabled by default as the springer template as of 2018/03/10 uses this as default

% German documents: pass ngerman as class option
% \documentclass[ngerman,runningheads,a4paper]{llncs}[2018/03/10]
% English documents: pass english as class option
\documentclass[english,runningheads,a4paper]{llncs}[2018/03/10]

%% If you need packages for other papers,
%% START COPYING HERE

% Set English as language and allow to write hyphenated"=words
%
% In case you write German, switch the parameters, so that the command becomes
%\usepackage[english,main=ngerman]{babel}
%
% Even though `american`, `english` and `USenglish` are synonyms for babel package (according to https://tex.stackexchange.com/questions/12775/babel-english-american-usenglish), the llncs document class is prepared to avoid the overriding of certain names (such as "Abstract." -> "Abstract" or "Fig." -> "Figure") when using `english`, but not when using the other 2.
% english has to go last to set it as default language
\usepackage[ngerman,main=english]{babel}
%
% Hint by http://tex.stackexchange.com/a/321066/9075 -> enable "= as dashes
\addto\extrasenglish{\languageshorthands{ngerman}\useshorthands{"}}
%
% Fix by https://tex.stackexchange.com/a/441701/9075
\usepackage{regexpatch}
\makeatletter
\edef\switcht@albion{%
  \relax\unexpanded\expandafter{\switcht@albion}%
}
\xpatchcmd*{\switcht@albion}{ \def}{\def}{}{}
\xpatchcmd{\switcht@albion}{\relax}{}{}{}
\edef\switcht@deutsch{%
  \relax\unexpanded\expandafter{\switcht@deutsch}%
}
\xpatchcmd*{\switcht@deutsch}{ \def}{\def}{}{}
\xpatchcmd{\switcht@deutsch}{\relax}{}{}{}
\edef\switcht@francais{%
  \relax\unexpanded\expandafter{\switcht@francais}%
}
\xpatchcmd*{\switcht@francais}{ \def}{\def}{}{}
\xpatchcmd{\switcht@francais}{\relax}{}{}{}
\makeatother

\usepackage{ifluatex}
\ifluatex
  \usepackage{fontspec}
  \usepackage[english]{selnolig}
\fi

\iftrue % use default-font
  \ifluatex
    % use the better (sharper, ...) Latin Modern variant of Computer Modern
    \setmainfont{Latin Modern Roman}
    \setsansfont{Latin Modern Sans}
    \setmonofont{Latin Modern Mono} % "variable=false"
    %\setmonofont{Latin Modern Mono Prop} % "variable=true"
  \else
    % better font, similar to the default springer font
    % cfr-lm is preferred over lmodern. Reasoning at http://tex.stackexchange.com/a/247543/9075
    \usepackage[%
      rm={oldstyle=false,proportional=true},%
      sf={oldstyle=false,proportional=true},%
      tt={oldstyle=false,proportional=true,variable=false},%
      qt=false%
    ]{cfr-lm}
  \fi
\else
  % In case more space is needed, it is accepted to use Times New Roman
  \ifluatex
    \setmainfont{TeX Gyre Termes}
    \setsansfont[Scale=.9]{TeX Gyre Heros}
    % newtxtt looks good with times, but no equivalent for lualatex found,
    % therefore tried to replace with inconsolata.
    % However, inconsolata does not look good in the context of LNCS ...
    %\setmonofont[StylisticSet={1,3},Scale=.9]{inconsolata}
    % ... thus, we use the good old Latin Modern Mono font for source code.
    \setmonofont{Latin Modern Mono} % "variable=false"
    %\setmonofont{Latin Modern Mono Prop} % "variable=true"
  \else
    % overwrite cmodern with the Times variant
    \usepackage{newtxtext}
    \usepackage{newtxmath}
    \usepackage[zerostyle=b,scaled=.9]{newtxtt}
  \fi
\fi

\ifluatex
\else
  % fontenc and inputenc are not required when using lualatex
  \usepackage[T1]{fontenc}
  \usepackage[utf8]{inputenc} %support umlauts in the input
\fi

\usepackage{graphicx}

% backticks (`) are rendered as such in verbatim environment. See https://tex.stackexchange.com/a/341057/9075 for details.
\usepackage{upquote}

% Nicer tables (\toprule, \midrule, \bottomrule - see example)
\usepackage{booktabs}

%extended enumerate, such as \begin{compactenum}
\usepackage{paralist}

%put figures inside a text
%\usepackage{picins}
%use
%\piccaptioninside
%\piccaption{...}
%\parpic[r]{\includegraphics ...}
%Text...

% For easy quotations: \enquote{text}
% This package is very smart when nesting is applied, otherwise textcmds (see below) provides a shorter command
\usepackage{csquotes}

% For even easier quotations: \qq{text}
\usepackage{textcmds}

%enable margin kerning
\RequirePackage[%
  babel,%
  final,%
  expansion=alltext,%
  protrusion=alltext-nott]{microtype}%
% \texttt{test -- test} keeps the "--" as "--" (and does not convert it to an en dash)
\DisableLigatures{encoding = T1, family = tt* }

%tweak \url{...}
\usepackage{url}
%\urlstyle{same}
%improve wrapping of URLs - hint by http://tex.stackexchange.com/a/10419/9075
\makeatletter
\g@addto@macro{\UrlBreaks}{\UrlOrds}
\makeatother
%nicer // - solution by http://tex.stackexchange.com/a/98470/9075
%DO NOT ACTIVATE -> prevents line breaks
%\makeatletter
%\def\Url@twoslashes{\mathchar`\/\@ifnextchar/{\kern-.2em}{}}
%\g@addto@macro\UrlSpecials{\do\/{\Url@twoslashes}}
%\makeatother

% Diagonal lines in a table - http://tex.stackexchange.com/questions/17745/diagonal-lines-in-table-cell
% Slashbox is not available in texlive (due to licensing) and also gives bad results. This, we use diagbox
%\usepackage{diagbox}

% Required for package pdfcomment later
\usepackage{xcolor}

% For listings
\usepackage{listings}
\lstset{%
  basicstyle=\ttfamily,%
  columns=fixed,%
  basewidth=.5em,%
  xleftmargin=0.5cm,%
  captionpos=b}%
\renewcommand{\lstlistingname}{List.}
% Fix counter as described at https://tex.stackexchange.com/a/28334/9075
\usepackage{chngcntr}
\AtBeginDocument{\counterwithout{lstlisting}{section}}

% Enable nice comments
\usepackage{pdfcomment}
%
\newcommand{\commentontext}[2]{\colorbox{yellow!60}{#1}\pdfcomment[color={0.234 0.867 0.211},hoffset=-6pt,voffset=10pt,opacity=0.5]{#2}}
\newcommand{\commentatside}[1]{\pdfcomment[color={0.045 0.278 0.643},icon=Note]{#1}}
%
% Compatibality with packages todo, easy-todo, todonotes
\newcommand{\todo}[1]{\commentatside{#1}}
% Compatiblity with package fixmetodonotes
\newcommand{\TODO}[1]{\commentatside{#1}}

% Bibliopgraphy enhancements
%  - enable \cite[prenote][]{ref}
%  - enable \cite{ref1,ref2}
% Alternative: \usepackage{cite}, which enables \cite{ref1, ref2} only (otherwise: Error message: "White space in argument")

% Doc: http://texdoc.net/natbib
\usepackage[%
  square,        % for square brackets
  comma,         % use commas as separators
  numbers,       % for numerical citations;
%  sort,          % orders multiple citations into the sequence in which they appear in the list of references;
  sort&compress, % as sort but in addition multiple numerical citations
                 % are compressed if possible (as 3-6, 15);
]{natbib}
% In the bibliography, references have to be formatted as 1., 2., ... not [1], [2], ...
\renewcommand{\bibnumfmt}[1]{#1.}

\ifluatex
  % does not work when using luatex
  % see: https://tex.stackexchange.com/q/419288/9075
\else
  % Prepare more space-saving rendering of the bibliography
  % Source: https://tex.stackexchange.com/a/280936/9075
  \SetExpansion
  [ context = sloppy,
    stretch = 30,
    shrink = 60,
    step = 5 ]
  { encoding = {OT1,T1,TS1} }
  { }
\fi

% Put footnotes below floats
% Source: https://tex.stackexchange.com/a/32993/9075
\usepackage{stfloats}
\fnbelowfloat

% Enable that parameters of \cref{}, \ref{}, \cite{}, ... are linked so that a reader can click on the number an jump to the target in the document
\usepackage{hyperref}
% Enable hyperref without colors and without bookmarks
\hypersetup{hidelinks,
  colorlinks=true,
  allcolors=black,
  pdfstartview=Fit,
  breaklinks=true}
%
% Enable correct jumping to figures when referencing
\usepackage[all]{hypcap}

\usepackage[group-four-digits,per-mode=fraction]{siunitx}

%enable \cref{...} and \Cref{...} instead of \ref: Type of reference included in the link
\usepackage[capitalise,nameinlink]{cleveref}
%Nice formats for \cref
\usepackage{iflang}
\IfLanguageName{ngerman}{
  \crefname{table}{Tab.}{Tab.}
  \Crefname{table}{Tabelle}{Tabellen}
  \crefname{figure}{\figurename}{\figurename}
  \Crefname{figure}{Abbildung}{Abbildungen}
  \crefname{equation}{Gleichung}{Gleichungen}
  \Crefname{equation}{Gleichung}{Gleichungen}
  \crefname{listing}{\lstlistingname}{\lstlistingname}
  \Crefname{listing}{Listing}{Listings}
  \crefname{section}{Abschnitt}{Abschnitte}
  \Crefname{section}{Abschnitt}{Abschnitte}
  \crefname{paragraph}{Abschnitt}{Abschnitte}
  \Crefname{paragraph}{Abschnitt}{Abschnitte}
  \crefname{subparagraph}{Abschnitt}{Abschnitte}
  \Crefname{subparagraph}{Abschnitt}{Abschnitte}
}{
  \crefname{section}{Sect.}{Sect.}
  \Crefname{section}{Section}{Sections}
  \crefname{listing}{\lstlistingname}{\lstlistingname}
  \Crefname{listing}{Listing}{Listings}
}


%Intermediate solution for hyperlinked refs. See https://tex.stackexchange.com/q/132420/9075 for more information.
\newcommand{\Vlabel}[1]{\label[line]{#1}\hypertarget{#1}{}}
\newcommand{\lref}[1]{\hyperlink{#1}{\FancyVerbLineautorefname~\ref*{#1}}}

\usepackage{xspace}
%\newcommand{\eg}{e.\,g.\xspace}
%\newcommand{\ie}{i.\,e.\xspace}
\newcommand{\eg}{e.\,g.,\ }
\newcommand{\ie}{i.\,e.,\ }

%introduce \powerset - hint by http://matheplanet.com/matheplanet/nuke/html/viewtopic.php?topic=136492&post_id=997377
\DeclareFontFamily{U}{MnSymbolC}{}
\DeclareSymbolFont{MnSyC}{U}{MnSymbolC}{m}{n}
\DeclareFontShape{U}{MnSymbolC}{m}{n}{
  <-6>    MnSymbolC5
  <6-7>   MnSymbolC6
  <7-8>   MnSymbolC7
  <8-9>   MnSymbolC8
  <9-10>  MnSymbolC9
  <10-12> MnSymbolC10
  <12->   MnSymbolC12%
}{}
\DeclareMathSymbol{\powerset}{\mathord}{MnSyC}{180}

\ifluatex
\else
  % Enable copy and paste - also of numbers
  % This has to be done instead of \usepackage{cmap}, because it does not work together with cfr-lm.
  % See: https://tex.stackexchange.com/a/430599/9075
  \input glyphtounicode
  \pdfgentounicode=1
\fi

% correct bad hyphenation here
\hyphenation{op-tical net-works semi-conduc-tor}

%% END COPYING HERE


% Add copyright
% Do that for the final version or if you send it to colleagues
\iffalse
  %state: intended|submitted|llncs
  %you can add "crop" if the paper should be cropped to the format Springer is publishing
  \usepackage[intended]{llncsconf}

  \conference{name of the conference}

  %in case of "llncs" (final version!)
  %example: llncs{Anonymous et al. (eds). \emph{Proceedings of the International Conference on \LaTeX-Hacks}, LNCS~42. Some Publisher, 2016.}{0042}
  \llncs{book editors and title}{0042} %% 0042 is the start page
\fi

% For demonstration purposes only
\usepackage[math]{blindtext}
\usepackage{mwe}
% ----------------------------------------------------------------------------------------------------------------------
% ----------------------------------------------------------------------------------------------------------------------
\begin{document}

\title{Zukünftige Entwicklungen in VANETs}
%If Title is too long, use \titlerunning
%\titlerunning{Short Title}

%Single insitute
\author{Fabian Sieber}
%If there are too many authors, use \authorrunning
%\authorrunning{First Author et al.}
\institute{Technische Universität Dortmund}

%% Multiple insitutes - ALTERNATIVE to the above
% \author{%
%     Firstname Lastname\inst{1} \and
%     Firstname Lastname\inst{2}
% }
%
%If there are too many authors, use \authorrunning
%  \authorrunning{First Author et al.}
%
%  \institute{
%      Insitute 1\\
%      \email{...}\and
%      Insitute 2\\
%      \email{...}
%}

\maketitle

\begin{abstract}
  Der Fortschritt in der Kommunikationstechnik, wie die Einführung von 5G, hat in den letzten Jahren zu einem wachsenden Interesse an Kommunikation mit und zwischen Fahrzeugen geführt.
  Dies beinhaltet sowohl Anwendungen im Bereich der Unterhaltung als auch im Bereich von Assistenzsystemen und der Verkehrsregelung.
  Aufgrund dieser Zunahme von Kapazitäten innerhalb eines Fahrzeugs bietet sich eine zusätzliche Verwendung der verbauten Technik an.
  Dazu wird in dieser Ausarbeitung das Konzept der \textit{Vehicular Clouds (VCs)} vorgestellt und zum herkömmlichen \textit{Cloud Computing} abgegrenzt.
  Außerdem werden mögliche Anwendungen vorgestellt, welche die Chancen beleuchten, die durch die Nutzung von \textit{Vehicular Clouds} entstehen.
  Sicherheit und Privatsphäre sind kritische Aspekte, auf die im späteren Verlauf dieser Ausarbeitung eingegangen wird.
  Zusätzlich wird die Aggregation von Ressourcen in einer \textit{Vehicular Cloud} betrachtet\cite{conti2013mobile}.
\end{abstract}

\begin{keywords}
  Mobile ad-hoc Netze, VANET, Cloud Computing, Vehicular Cloud
\end{keywords}
% ----------------------------------------------------------------------------------------------------------------------
\section{Einleitung}\label{sec:intro}
  Der Fortschritt in der Kommunikationstechnik, wie die Einführung von 5G, hat in den letzten Jahren zu einem wachsenden Interesse an Kommunikation mit und zwischen Fahrzeugen geführt.
  Anfängliche Vorstellungen der zukünftigen Entwicklungen versprechen eine verstärkte Kommunikation zwischen den Fahrzeugen, welche durch den Austausch von Informationen über die Strassenverhältnisse oder andere verkehrsbezogene Gefahren zu einem sichereren Verkehr auf unseren Autobahnen und Straßen führt.
  Diese Aussicht führte zu einer vermehrten Ausstattung von Fahrzeugen mit Kommuniationstechnik und der Einführung von Verkehrstelematik (\textit{Intelligent Transportations Systems ITS})\cite{conti2013mobile}.
  Ein Beispiel dafür sind an Autobahnen montierte Wechselverkehrszeichen, welche über Straßenverhältnisse informieren, flexible Geschwindigkeitsbegrenzungen ermöglichen oder auch die Öffnung und Schließung von Fahrspuren ankündigen.
  Mithilfe dieser können beispielsweise bei einem Unfall die betroffenen Spuren gesperrt werden, um die Unfallstelle, besonders bei nicht vorhandener Geschwindigkeitsbegrenzung, frühzeitig zu sichern, oder auch in Hauptverkehrszeiten der Standstreifen als zusätzliche Fahrspur freigegeben werden.

  In den Vereinigten Staaten von Amerika wurde durch die \textit{US Federal Communications Commission (FCC)} ein 75 MHz breites Spektrum im Bereich des 5.850- bis 5.925-GHz-Band für die Nutzung durch \textit{Dedicated Short-Range Communications (DSRC)} reserviert.
  Dies übertrifft die für die verkehrssichernde Kommunikation notwendigen Kapazitäten.
  Deshalb wird erwartet, dass Drittanbieter diese überflüssige Bandbreite nutzen, um eigene Anwendungen zu schaffen, welche den Verkehrsteilnehmern angeboten werden.
  Umgesetzt wurde dies bereits jetzt in Ortsbezogenen Anwendungen oder auch Unterhaltungsmedien.
  Ein Beispiel dafür ist die Integration von Spotify, YouTube und Netflix oder auch Videospielen in Fahrzeugen des Herstellers Tesla.
  Mit der Verbreitung dieser Anwendungen wird von dem Ausbau der Infrastruktur entlang der Verkehrswege durch Drittanbieter und die Ausstattung von Fahrzeugen mit fähigerer Technik ausgegangen\cite{conti2013mobile}.

  Der aktuelle Stand der Technik in Fahrzeugen reagiert lediglich auf Ereignisse, welche durch die eigene Sensorik wahrgenommen werden und kann nur auf das eigene Fahrzeug einwirken.
  Ziel ist es, dass diese Informationen und Ressourcen zwischen den Fahrzeugen geteilt werden, um gemeinsam ein Problem zu lösen, welches ohne Kooperation übermäßig lang zu lösen dauern würde (beispielsweise die Auflösung eines Staus) oder mit den begrenzten Ressourcen eines einzelnen Fahrzeugs nicht lösbar wäre.
  Diese Ausarbeitung stellt das Konzept der \textit{Vehicular Clouds (VCs)} vor, welches die zwischen den Fahrzeugen bereits vorhandenen Netze ``in die Cloud bringt''\cite{conti2013mobile}.
  Dazu wird zunächst in Kapitel \ref{sec:basics} auf einige Grundlagen eingegangen, welche das Konzept vorstellen und dafür nötiges Wissen vermitteln.
  Daraufhin wird in Kapitel \ref{sec:ccvsvc} die \textit{Vehicular Cloud} vom herkömmlichen Begriff \textit{Cloud Computing} abgegrenzt, um die Notwendigkeit von Neuerungen und die entstehenden Herausforderungen zu verdeutlichen.
  In Kapitel \ref{sec:examples} werden zur Darstellung der durch \textit{Vehicular Clouds} entstehenden Möglichkeiten Anwendungsbeispiele vorgestellt.
  Aufgrund der Verbindung der Systeme innerhalb eines Fahrzeugs und der Masse an Daten sind Sicherheit und Privatsphäre besonders wichtig.
  Diese Bedenken werden in Kapitel \ref{sec:securityprivacy} behandelt.
  Im Anschluss befasst sich Kapitel \ref{sec:aggregation} mit der Art und Weise, wie die Ressourcen in \textit{Vehicular Clouds} verteilt und erlangt werden können.
  Schließlich wird in Kapitel \ref{sec:conclusion} ein Fazit gezogen und ein Ausblick auf weitere Entwicklungen gegeben.
% ----------------------------------------------------------------------------------------------------------------------
\section{Grundlagen}
\label{sec:basics}
In diesem Kapitel wird zunächst auf die bisher verbaute Technik in Fahrzeugen eingegangen und welche Entwicklungen dabei in naher Zukunft absehbar sind.
Des Weiteren werden Fahrzeug-Ad-hoc-Netze vorgestellt, auf denen \textit{Vehicular Clouds} grundlegend basieren.
Außerdem wird auf \textit{Cloud Computing} eingegangen, da darin ebenfalls eine Grundlage besteht und dies im weiteren Verlauf zu \textit{VCs} abgegrenzt werden soll.
Schließlich wird das Konzept der \textit{VCs} vorgestellt.

\subsection{Verbaute Technik}
In den letzten zwei Jahrzenten hat sich ein Trend der Fahrzeughersteller abgezeichnet, ihre Fahrzeuge intelligenter und sicherer zu gestalten.
Das Fahrerlebnis soll dadurch sicherer, stressfreier und schließlich angenehmer für ihre Kunden werden.
Deshalb ist heutzutage in einem Fahrzeug oft mindestens eines der folgenden Geräte zu finden:
Ein Bordcomputer, ein GPS-Peilsender oder Navigationsgerät, eine Parkdistanzkontrolle beziehungsweise ein Parkassistent oder auch ein kamerabasiertes System.
Ebenso befinden sich in einigen Oberklassemodellen bereits sogenannte \textit{Blackboxes}, welche ähnlich zu dem meist bekannten Einsatz in Flugzeugen eine Vielzahl von Fahrzeugdaten, wie die Umdrehungszahl des Motors oder auch ABS-Signale, speichern.
Auf Grundlage dieser Daten lassen sich beispielsweise Rückschlüsse auf das Fahrverhalten des Fahrers oder auch die Straßenbedingungen vor einem Unfall schließen\cite{conti2013mobile}.

Im Jahr 2015 erließ die Europäische Union eine Verordnung, welche die Fahrzeughersteller dazu verpflichtet, ab 2018 ein sogenanntes eCall-System zu verbauen, welches zum Absetzen eines Notrufs im Falle eines Unfalls dient.
Dies gilt für alle neuen Fahrzeugmodelle, welche ab 2018 ihre Typgenehmigung erhalten sollen, um somit auf dem Europäischen Markt verkauft und angemeldet werden zu dürfen\cite{euECall}.
Ebenso testete der Allgemeine Deutsche Automobil-Club (ADAC) im Jahr 2020 zwei Fahrzeugmodelle unterschiedlicher Hersteller hinsichtlich des Sammelns von Daten.
Bei einer Mercedes B-Klasse mit der Typbezeichnung W246 (2011-2018)  wurde festgestellt, dass ungefähr alle zwei Minuten die GPS-Position des Fahrzeugs, Kilometerstand, Tankfüllung, Reifendruck und weitere Informationen an Mercedes übertragen werden.
Zusätzlich verfügt das Fahrzeug über einen elektromotorischen Gurtstraffer, welcher zum Beispiel bei starken Bremsungen auslöst, um die Insassen zu fixieren.
Die Anzahl der Auslösungen des Gurtstraffers werden gespeichert und lassen somit Rückschlüsse auf die Fahrweise zu.
Ebenso werden Fehlerspeicher-Einträge mit Informationen über die Motordrehzahl und Motortemperatur abgelegt, um Rückschlüsse auf die Fahrweise zu geben.
Außerdem werden die Daten in unterschiedlichen Fahrprofilen für Autobahn-, Überland-, und Stadtfahrten gespeichert und durch weitere Aufzeichnungen Fahr- und Standzeiten festgehalten\cite{adacDaten}.
Das Speichern dieser Informationen und besonders der Versand von Teilen dieser Informationen an Dritte birgt die Frage, wie sicher und privat diese Daten sind und ob sie eventuell missbraucht werden.

Aufgrund der Zunahme an Sensorik und Datenverarbeitung wachsen auch die Ansprüche an die verbaute Technik und somit die verfügbare Rechenleistung der Fahrzeuge.
Ebenso verbreiten sich Fahrzeuge mit Elektro- und Hybriden Antrieben stetig weiter.
Dadurch werden neue Möglichkeiten eröffnet, da solche Fahrzeuge im Vergleich zu Verbrennern über immens größere Batteriekapazitäten verfügen und somit über einen längeren Zeitraum ihre Rechenleistung bereitstellen können\cite{conti2013mobile}.

\subsection{Fahrzeug-Ad-hoc-Netze}
Das Auftreten von Stau hat sich in den letzten Jahren besonders in Ballungsgebieten zu einem stetigen Problem entwickelt.
Stau verschwendet allerdings nicht nur unsere Zeit, sondern sorgt im Großen und Ganzen für einen vermehrten \(CO_2\)-Ausstoß und besonders in Großstädten für eine übermäßige Feinstaubbelastung.
Um einen Stau aufzulösen oder auch zu verhindern, wird heutzutage auf Kameras, Induktionsschleifen in der Fahrbahn (\textit{ILDs}) und Radarsensoren zurückgegriffen.
Diese Hilfsmittel sind allerdings kostenintensiv, an den Ort gebunden und ihr Nutzen eingeschränkt.
Die Effizienz der Erkennung von Verkehrsereignissen lässt sich allerdings durch den Stand der Technik anderer Gebiete verbessern.
Dadurch ergab sich eine Abwandlung von Mobilen Ad-hoc-Netzen (\textit{MANETs}) im Bereich der Kommunikation im Straßenverkehr, genannt Fahrzeug-Ad-hoc-Netze (\textit{VANETs}).

In Fahrzeug-Ad-hoc-Netzen wird zwischen zwei Arten von Kommunikation unterschieden.
Die Fahrzeug-zu-Fahrzeug-Kommunikation (\textit{V2V}) setzt darauf, dass Fahrzeug ihre wahrgenommen Verkehrsereignisse mit anderen Fahrzeugen teilen und ebenso auch auf die Informationen anderer Fahrzeuge reagieren.
Daraus ergeben sich allerdings zusätzliche Sicherheitsprobleme, da gezielte Fehlinformationen zu einer Verschlechterung der Verkehrssituation oder sogar zu Unfällen führen könnten\cite{conti2013mobile}.
Anfang 2020 ereignete sich beispielsweise ein Fall, in dem ein Künstler mit 99 Smartphones mehrfach eine Straße entlang lief und somit auf Google Maps einen vermeintlichen Stau erzeugte, der gar nicht existierte\cite{googleMaps2020}.
In so einem Fall könnte durch die Umfahrung des vermeintlichen Staus auf den umliegenden Straßen aufgrund des erhöhten Verkehrsaufkommens ein tatsächlicher Stau auftreten.
Die Fahrzeug-zu-Infrastruktur-Kommunikation (\textit{V2I}) sorgt für den Austausch von Informationen zwischen den Fahrzeugen und an den Straßen montierten Geräten.
Dies umfasst beispielsweise die Weitergabe der Anzahl von Fahrzeugen, die an einer vorausliegenden Ampel halten und durch eine fest montierte Kamera erfasst wurden.
Fahrzeug-Ad-hoc-Netze befinden sich weiterhin auf dem Vormarsch der Verbreitung und Weiterentwicklung.
Diese Entwicklungen lassen uns auf eine Revolution des Straßenverkehrs hoffen, sodass eine sichere, robuste und mit einer Rechnerallgegenwart versehene Umgebung innerhalb unseres Straßenverkehrs entsteht\cite{conti2013mobile}.

\subsection{Cloud Computing}
\label{sec:cc}
Im Allgemeinen wird der Begriff \textit{Cloud Computing} als Synonym für über das Internet bereitgestellte Gehostete Dienste genutzt.
Genauer genommen wird damit ein Modell bezeichnet, welches Ressourcen zeitnah und ohne Eingriff des Bereitstellers zur Verfügung stellt.
Solche Ressourcen sind beispielsweise Bandbreite, Arbeitsspeicher, Rechenleistung oder auch Speicher.
\textit{Cloud Computing} hat sich als eigenes Geschäftsfeld etabliert, da mittlerweile ein relativ günstiger Highspeed-Internetzugang möglich ist und die aktuelle Technik eine effiziente Virtualisierung ermöglicht.
Dadurch ist es besonders für kleine Unternehmen oft wirtschaftlich, auf Drittanbieter zurückzugreifen, anstatt eigene Infrastruktur aufzubauen.
Dies erübrigt zusätzlich die Notwendigkeit von Angestellten, welche die eigene Infrastruktur bedienen und warten können müssten.
\textit{Cloud IT services} erweitern dieses Modell um Anwendungen und Dienste, sodass Nutzer mit minimalem Aufwand ihre benötigten Ressourcen skalieren können.
Dadurch ergeben sich drei schwerwiegende Vorteile\cite{conti2013mobile}:
\begin{itemize}
\item Nutzer haben den Eindruck von unendlichen Ressourcen, die sie bei Bedarf abrufen können, sodass sich weite Vorausplanung erübrigt.
\item Nutzer haben die Möglichkeit mit wenigen Ressourcen zu beginnen und diese bei Bedarf einfach erweitern zu können.
\item Nutzer können Ressourcen für kurze Zeiträume buchen und somit in der Zwischenzeit kosten sparen, falls große Mengen von Ressourcen nur zeitweise benötigt werden.
\end{itemize}

\begin{figure}[h]
\centering
\includegraphics[width=0.7\textwidth]{images/cloudComputing}
\caption{Unterschiede zwischen den Kategorien des \textit{Cloud Computing}\cite{azure}}
\label{fig:ccCategories}
\end{figure}

Wie in Abbildung \ref{fig:ccCategories} gezeigt lässt sich \textit{Cloud Computing} in folgende drei Kategorien unterteilen\cite{conti2013mobile}:
\begin{itemize}
\item[1.] \textit{Infrastructure as a Service (IaaS)}. Dem Nutzer werden Ressourcen wie Rechenleistung, Speicher und Bandbreite bereitgestellt. Ein Beispiel dafür sind die \textit{Amazon Web Services (AWS)}.
\item[2.] \textit{Platform as a Service (PaaS)}. Dem Nutzer werden zusätzlich Entwicklerwerkzeuge zur Verfügung gestellt, um Webanwendungen ohne lokale Installation entwickeln und betreiben zu können. \textit{Microsoft Azure} ist ein Beispiel für diese Kategorie.
\item[3.] \textit{Software as a Service (SaaS)}. Dem Nutzer wird die Verwendung einer Anwendung per Abonnement gestattet, sodass kostspielige Software zeitweise genutzt werden kann. Dies kann günstiger als eine Eigenentwicklung oder auch ein permanenter Erwerb der Software sein. Ein Beispiel für diese Kategorie sind \textit{Microsoft 365} oder auch die \textit{Adobe Creative Cloud}.
\end{itemize}

\begin{figure}[h]
\centering
\includegraphics[width=0.7\textwidth]{images/cloudArchitecture}
\caption{Die Eucalyptus Cloud Architektur.}
\label{fig:ccArchitecture}
\end{figure}

Abbildung \ref{fig:ccArchitecture} stellt den Aufbau der Open-Souce-Infrastruktur Eucalyptus und somit die Funktionsweise von \textit{Cloud Computing} dar.
Der Nutzer kommuniziert mittels einer Anwendung mit dem \textit{Cloud Controller}, welcher den Austausch von Daten mit mehreren Diensten ermöglicht, die vom Back-End bereitgestellt werden.
Dazu kommuniziert der \textit{Cloud Controller} mit mehreren \textit{Cluster Controllern}, welche wiederum mit mehreren \textit{Node Controllern} kommunizieren.
Im Normalfall ist solch ein \textit{Node Controller} eine Anwendung, die auf einem einzelnen Server eines Rechenzentrums läuft\cite{conti2013mobile}.

\subsection{Vehicular Clouds}
\label{sec:vc}
Kapitel \ref{sec:cc} hat gezeigt, dass durch \textit{Cloud Computing} weniger ungenutzte Ressourcen entstehen und die zugrundeliegenden Infrastrukturen somit wirtschaftlicher werden.
Zusätzlich ist bekannt, dass Fahrzeuge die meiste Zeit ungenutzt auf Parkplätzen, in Garagen oder am Straßenrand stehen und somit ihre Ressourcen verschwendet werden.
Dadurch erscheint es logisch, dass das Interesse entsteht, dieses Prinzip auch auf Fahrzeuge anzuwenden.
Ein Fahrzeug würde im Zusammenhang des \textit{Cloud Computing} also einen Knoten (\textit{Node}) darstellen.
Allerdings unterscheiden sich Fahrzeuge zu konventionellen Knoten insofern, dass sie nicht nur stehen, sondern sich auch bewegen können und daher über keinen festen Zugang zum Netzwerk und einer Stromversorgung verfügen.
Beispielsweise müsste ein Fahrzeug bei der Verwendung des 5G Netzwerks während der Fahrt mehrfach den Sendemasten wechseln, um weiterhin kommunizieren zu können.

Es wird davon ausgegangen, dass Fahrzeugbesitzer durch eine Entschädigung dazu angespornt werden könnten, die Ressourcen ihres Fahrzeugs zur Verfügung zu stellen.
Besonders im Falle eines Staus wären Fahrzeugbesitzer sicherlich dazu bereit die Ressourcen ihres Fahrzeugs zur Auflösung des Staus bereitzustellen.
Mit diesen Ressourcen könnten Berechnungen zur Auflösung eines Staus angestellt werden, die mithilfe eines zentralen Systems nicht rechtzeitig berechnet werden könnten.
% ----------------------------------------------------------------------------------------------------------------------
\section{Cloud Computing und Vehicular Clouds: Worin liegt der Unterschied?}
\label{sec:ccvsvc}
Oberflächlich betrachtet weichen statische \textit{Vehicular Clouds} nicht vom herkömmlichen \textit{Cloud Computing} ab.
Genauer betrachtet existieren allerdings zusätzliche Herausforderungen, wie beispielsweise eine limitierte Stromversorgung.
Außerem können, wie in Kapitel \ref{sec:vc} beschrieben, auch dynamische \textit{VCs} existieren, in denen die Fahrzeuge sich bewegen.
Fahrzeuge werden währenddessen vermehrt auf Probleme treffen, die aufgrund ihres spontanen Auftretens nicht mit vordefinierten Ressourcen gelöst werden können.
Dabei häufen sich meist mehrere Fahrzeuge, beispielsweise in Form eines Staus.
Es muss eine zeitnahe Einschätzung der benötigten Ressourcen erfolgen und durch die Fahrzeuge vor Ort bereitgestellt werden.
Diese Fähigkeit besitzt \textit{Cloud Computing} nicht.
Im Folgenden werden drei zusätzliche Cloud-Dienste vorgestellt, die durch \textit{VCs} realisierbar sind\cite{conti2013mobile}.

\subsection{\textit{Network as a Service (NaaS)}}
Während manche Fahrzeuge während der Fahrt über eine Internetverbindung verfügen, muss dies nicht für alle Fahrzeuge der Fall sein.
Es bietet sich für zuerst genannte Fahrzeuge an, ihre Internetverbindung den anderen Fahrzeugen zur Verfügung zu stellen.
Vergleicht man diesen Anwendungsfall mit der Nutzung von Smartphones, wird einem klar, dass in der Zukunft vermutlich sämtliche Fahrzeuge über eine Internetverbindung per Mobilfunk oder WiFi verfügen werden.
Doch selbst dann bietet sich dieser Dienst an, da Fahrzeuge ihre überschüssige Bandbreite zur Verfügung stellen können, falls diese von anderen Fahrzeugen für beispielsweise Unterhaltungsanwendungen benötgt wird.
Es wird davon ausgegangen, dass die Bereitschaft die Bandbreite zur Verfügung zu stellen durch die jeweiligen Fahrzeuge verbreitet wird.
Diese Information könnte von anderen Fahrzeugen transient weitergegeben werden, um sich als Knoten zwischen dem bereitstellenden und dem anfordernden Fahrzeug anzubieten.
Dadurch könnten auch Fahrzeuge diesen Dienst nutzen, die sich nicht in direkter Umgebung eines Fahrzeugs mit Mobilfunk befinden.
Einfacher wird dieser Dienst sogar durch die Tatsache, dass die Fahrzeuge im Durchschnitt nur geringe Geschwindigkeitsunterschiede besitzen werden, solange sie dieselbe Fahrtrichtung besitzen.
Deshalb kann das System auf Protokolle und Erkenntnisse traditioneller Mobiler Ad-Hoc-Netzwerke \textit{(MANETs)} zurückgreifen\cite{conti2013mobile}.

\begin{figure}[h]
\centering
\includegraphics[width=0.7\textwidth]{images/networkAAS}
\caption{Illustration einer NaaS-Anwendung.}
\label{fig:NaaS}
\end{figure}

Abbildung \ref{fig:NaaS} stellt solch eine Situation auf einer dreispurigen Autobahn beispielhaft dar.
Neben der Autobahn befindet sich ein Zugangspunkt \textbf{w}, welcher mittels WiFi eine Internetverbindung bereitstellt.
Das Fahrzeug \textbf{e} befindet sich in direkter Nähe zum Zugangspunkt \textbf{w} und kann somit diese Internetverbindung transient zur Verfügung stellen.
Außerdem verfügen die Fahrzeuge \textbf{a} und \textbf{d} über 5G.
Die Fahrzeuge \textbf{a}, \textbf{d} und \textbf{e} werden diese Information regelmäßig an die Fahrzeuge in ihrer Nähe senden.
Sollte sich nun also beispielsweise Fahrzeug \textbf{c} dazu entscheiden, eine Internetverbindung herstellen zu wollen, kann es anhand mehrerer Parameter abschätzen, welches Angebot am besten zu seinem Anwendungszweck passt.
Solche Parameter sind beispielsweise die relative Geschwindigkeit zwischen den Fahrzeugen, die Stabilität der Verbindung, die Verbindungsgeschwindigkeit oder auch die voraussichtliche Nutzungsdauer der Verbindung.
Falls Fahrzeug \textbf{c} eine kleine Nachricht, wie eine Mail, versenden möchte oder auch eine kleine Datei herunterladen möchte, wäre es sinnvoll die Verbindung über Fahrzeug \textbf{e} zu dem Zugangspunkt \textbf{w} zu nutzen, da eine hohe Verbindungsgeschwindigkeit und eine voraussichtlich niedrige Verbindungsdauer vorliegen.
Ist allerdings die Stabilität der Verbindung essenziell oder die voraussichtliche Dauer der Verbindung länger, wie bei einem Videotelefonat, bietet sich eine Verbindung zu Fahrzeug \textbf{a} oder \textbf{d} eher an\cite{conti2013mobile}.

\subsection{\textit{Cooperation as a Service (CaaS)}}
Fahrzeug-Ad-hoc-Netze wurden entwickelt, um den Verkehrsteilnehmern ein breites Spektrum an Diensten zur Verfügung zu stellen.
Darunter fallen Verkehrsinformationen, Wetter- und Straßenverhältnisse, Verkehrswarnungen oder auch Erweiterungen von Assistenzsystemen, die der Sicherheit der Verkehrsteilnehmer dienen.
Diese Dienste können mittels 5G oder Verkehrstelematik \textit{(ITS)} bereitgestellt werden.
Dazu werden allerdings Systeme an den Straßen und Autobahnen benötigt, welche sowohl mit anfänglichen Baukosten als auch mit dauerhaften Betriebskosten verbunden sind.
Außerdem benötigt jedes Fahrzeug dazu ein mindestmaß an Bandbreite, wodurch ein großer Ressourcenverbrauch entstehen kann.
Um die so entstehenden Kosten gering zu halten, wurde das Konzept \textit{Cooperation as a Service (CaaS)} entwickelt.
\textit{CaaS} setzt auf ein Publish-Subscribe-Muster, in dem Absender und Empfänger von Nachrichten ihr Interesse an Diensten mitteilen und daraufhin bei überschneidenden Interessen kooperieren, um alle für diesen Dienst notwendigen Informationen zu sammeln und auszutauschen.
Das Netzwerk könnte dazu in mehrere \textit{Cluster}, welche respektiv für die Dienste sind, aufgeteilt werden.
Für die Kommunikation zwischen den \textit{Clustern} wird \textit{Delay Tolerant Networking} genutzt, um nicht durch eine geringere Stabilität des Netzwerks aufgrund der neuen Herausforderungen von \textit{VCs} eingeschränkt zu werden.
Für die Kommunikation innerhalb von \textit{Clustern} wird auf \textit{content-based routing} gesetzt, sodass Nachrichten mittels Überprüfung des Inhalts an die Fahrzeuge weitergeleitet werden können, die sich dafür interessieren, ohne zuvor explizit die Empfänger festzulegen\cite{conti2013mobile}.

\subsection{\textit{Storage as a Service (STaaS)}}
Eine weitere Idee besteht darin, dass Fahrzeuge auch ihren Speicherplatz zur Verfügung stellen könnten.
Dabei muss allerdings beachtet werden, dass die Nutzer dieses Dienstes, im Gegensatz zu \textit{NaaS} oder \textit{IaaS}, eventuell keinen sofortigen Vorteil erhalten, sondern dieser erst nach einer bestimmten Zeitspanne eintritt.
Dies trifft beispielsweise auf die Auslagerung von Backups zu.
Es wird ebenso davon ausgegangen, dass Fahrzeuge aufgrund von zunehmend geringeren Kosten und geringerer Baugröße von Speicher die Fahrzeuge mit mehr Speicher ausgestattet werden sein, als notwendig wäre\cite{conti2013mobile}.
Diesem Punkt kann ich allerdings nicht vollkommen zustimmen, da in der Automobilbranche bei Zulieferern bereits geringe Preisunterschiede ausschlaggebend sind.
Deshalb werden aufgrund der Produktionsmenge die Fahrzeuge vermutlich nicht mit übermäßigen Speicherkapazitäten ausgestattet sein.
Eine zusätzliche Herausforderung besteht darin, dass die Fahrzeuge sich vermutlich voneinander entfernen, bevor der Nutzen des Dienstes eingetreten ist.
Je nach Anwendungsfall könnte allerdings selbst dann ein Zugriff auf die gespeicherten Informationen per Internet erfolgen.
Es lassen sich also sicherlich Anwendungsfälle für diesen Dienst finden, welche trotz dieser Herausforderung einen Nutzen bieten.

% ----------------------------------------------------------------------------------------------------------------------
\section{Anwendungsbeispiele}
\label{sec:examples}
In diesem Kapitel werden Anwendungsbeispiele vorgestellt, um neue Möglichkeiten und Lösungen für bestehende Probleme mit \textit{Vehicular Clouds} zu verdeutlichen.
Dazu werden sowohl Anwendungen mit stillstehenden Fahrzeugen, welche vermutlich einfacher umzusetzen sind, als auch Anwendungen mit Fahrzeugen in Bewegung betrachtet.

\subsection{Das Flughafenparkhaus als Rechenzentrum}
\label{sec:flughafen}
Der Flughafen Düsseldorf verfügt über 18415 Parkplätze, von denen 8589 Langzeit-Parkplätze sind\cite{flugDuess}.
Gehen wir nun lediglich von den Langzeit-Parkplätzen und einer durchschnittliche Belegung von ungefähr der Hälfte aus.
Daraus ergibt sich, dass ungefähr 10000 Fahrzeuge ungenutzt am Flughafen herumstehen, während ihre Besitzer auf Reisen sind.
Außerdem können wir davon ausgehen, dass fast jedes Fahrzeug aufgrund der Reisedauer mehrere Tage dort stehen wird.
Da Abflug und Ankunft der Reisenden im Normalfall im Voraus geplant sind, lassen sich sogar die Einbindung und Trennung der einzelnen Fahrzeuge von der \textit{VC} vergleichsweise einfach organisieren.
Mit zunehmenden Verkaufszahlen von Elektro- und Hybridfahrzeugen verbreitet sich auch die Ladeinfrastruktur.
Beispielsweise verfügt der Flughafen Düsseldorf bereits in 3 der 11 Parkhäuser über Ladesäulen\cite{flugDuess}.

Die Idee besteht darin, dass die geparkten Fahrzeuge durch den Flughafen mit Strom und einer Internetverbindung versorgt werden.
Im Austausch gegen kostenloses Parken erlaubt der Fahrzeugbesitzer die Teilnahme seines Fahrzeugs an der \textit{VC}.
Hinsichtlich der üblich relativ hohen Kosten von Parkplätzen an Flughäfen und einer vollgeladenen Batterie bei Rückkehr des Fahrzeugbesitzers, im Falle von Elektrofahrzeugen, erscheint die Partizipation an der \textit{VC} sehr wahrscheinlich.
Die Motivation der Flughäfen besteht darin, dass der Betrieb solch eines Rechenzentrums vermutlich wirtschaftlicher als der bisherige Betrieb der Parkhäuser wäre.
Es erscheint daher logisch, dass in Zukunft die Ressourcen der Fahrzeuge gesammelt und für die Zwecke des Flughafens genutzt oder auch an Dritte vermietet werden\cite{conti2013mobile}.

\subsection{Dynamische Regelung des Verkehrs}
Besonders bei Sport- und Kulturveranstaltung besteht spätestens bei Ende der Veranstaltung ein stark erhöhtes Verkehrsaufkommen.
Rund um den Signal Iduna Park befinden sich beispielsweise ungefähr 10000 Parkplätze, welche teilweise ebenfalls als Parkmöglichkeiten für die sich daneben befindenden Messehallen dienen.
Zusätzlich existieren an der TU Dortmund weitere Parkplätze, die mittels Shuttle-Bus erreichbar sind.
Durchschnittlich alle zwei Wochen bei einem Heimspiel des Vereins Borussia Dortmund, zu dem über 80000 Zuschauer anreisen (ein Großteil per öffentlichem Nahverkehr), kommt es bereits Stunden vor Anpfiff zu einem erhöhten Verkehrsaufkommen in der Dortmunder Innenstadt und der direkten Umgebung des Stadions.
Besonders nach Ende des Spiels ist das Verkehrsaufkommen so hoch, dass sowohl die anliegenden Bundesstraßen B1 und B54 als auch umliegende Straßen stundenlang durch Stau beeinträchtigt sind.
Die Auflösung dieser Staus ist im Sinne aller Beteiligten, weshalb von der Bereitschaft der Teilnahme an einer \textit{VC} zur Lösung des Problems ausgegangen wird.

Das Problem ist allerdings dynamisch und so komplex, dass eine herkömmliche Berechnung undenkbar ist.
Bis diese erfolgt ist, kann die Situation sich bereits so verändert haben, dass die Berechnungen mittlerweile unbrauchbar geworden sind.
Außerdem können diese Staus an den betroffenen Stellen in unterschiedlichem Maße auftreten, sodass eine vorherige Einschätzung der benötigten Ressourcen schwierig ist.
Mit der \textit{VC} steht eine große Menge von Ressourcen in direkter Nähe des Problems bereit, welche im benötigten Maße zugewiesen werden kann und sofort bereitsteht.
Ebenso können von den Fahrzeugen wahrgenommene Veränderungen der Situation direkt in die Berechnungen einfließen, um eine Ressourcenverschwendung zu verhindern\cite{conti2013mobile}.

Dieses Beispiel lässt sich allerdings auch auf den Alltag anwenden.
Jeden Tag kommt es zu den Hauptverkehrszeiten, aufgrund von ähnlichen Arbeitszeiten, zu einem erhöhten Verkehrsaufkommen oder auch Staus auf unseren Straßen.
Durch eine dynamische Regelung von Ampeln und Verkehrstelematik könnten diese mithilfe von Berechnungen und Sensordaten der Fahrzeuge in einer \textit{VC} aufgelöst oder zumindest vermindert werden.
Dies wird besonders dadurch begünstig, dass die Fahrzeuge das Problem eigenständig lösen und sich nicht auf die Ressourcen Dritter verlassen und warten müssen\cite{conti2013mobile}.

\subsection{Austausch von Informationen während der Fahrt}
Ein weiterer Anwendungsfall besteht in dem Austausch von Informationen wie Straßenverhältnissen oder auch plötzlich auftretenden Ereignissen.
Heutige Fahrzeuge verfügen über eine zunehmende Menge von Sensoren wie Kameras, \textit{LiDAR} oder auch Ultraschallsensoren, welche Assistenzsystemen zur Verbesserung der Fahrzeugsicherheit dienen.
Deshalb kann es nützlich sein, wenn diese Informationen zwischen den Fahrzeugen ohne zusätzliche Infrastruktur geteilt werden würden\cite{conti2013mobile}.
Ein Beispiel dafür ist das Auftreten eines Staus auf Autobahnen.
Besonders bei schlechten Sichtverhältnissen übersehen Fahrer oft das Ende eines Staus, sodass es zu einem Auffahrunfall kommt.
Dies könnte verhindert werden, indem die stehenden Fahrzeuge diese Information an alle anderen verteilen oder auch die Fahrzeuge, welche das Stauende wahrnehmen, die anderen Fahrzeuge über das vorausliegende Stauende warnen.
Es muss allerdings nicht immer ein Stau sein, das Bremsen eines vorausfahrenden Fahrzeugs reicht bereits aus.
Aktuell erkennt jedes Fahrzeug einzeln den Bremsvorgang des direkt vorausfahrenden Fahrzeugs und löst dann den Bremsassistenten aus.
Diese Information könnte nun auf einfache Weise mit den Fahrzeugen dahinter geteilt werden, sodass diese bereits bei der Auslösung des Bremsassistenten des Vorausfahrenden reagieren können und dies nicht erst wahrnehmen müssen.
Dadurch ließen sich besonders bei hohen Geschwindgikeiten viele Auffahrunfälle und somit auch dadurch entstehende Staus verhindern.

Selbiges gilt für Informationen, die gar nicht oder zu spät durch das eigene Fahrzeuge erfasst werden können.
Ist die Sicht auf ein herausziehendes Fahrzeug auf der Autobahn beispielsweise durch einen Kleintransporter auf der Mittelspur verdeckt, kann es zu einem gravierenden Unfall bei hoher Geschwindigkeit kommen.
Diese Information kann sowohl durch das herausziehende Fahrzeug als auch durch den Kleintransporter, der diesen Spurwechsel wahrnimmt, bereitgestellt werden.
Generell kann die Verwendung von \textit{VCs} den Verkehr auf unseren Straßen sicherer gestalten.

% ----------------------------------------------------------------------------------------------------------------------
\section{Sicherheit und Privatsphäre}
\label{sec:securityprivacy}

Die Nutzung einer Menge von Ressourcen durch mehrere Nutzer weckt im Besonderen zwei Bedenken, Sicherheit und Privatsphäre.
Auf der einen Seite müssen die Sicherheit und Privatsphäre des Fahrzeuges und Fahrzeugsbesitzers gewahrt werden, welche die Ressourcen bereitstellen.
Andererseits müssen auch die Sicherheit und Privatsphäre der Nutzer gewahrt werden.
Dabei sind viele Herausforderungen und Lösungsmöglichkeiten aus anderen Netzen übertragbar.
\textit{VCs} besitzen allerdings zusätzliche Herausforderungen, die beispielsweise aus der Mobilität der Fahrzeuge hervorgehen\cite{conti2013mobile}:
\begin{itemize}
\item Die Authentifizierung von Fahrzeugen und ihren Besitzern beziehungsweise Fahrern.
\item Der Aufbau von Vertrauensbeziehungen zwischen den Knoten des Netzes.
\item Die Skalierbarkeit der Sicherheitsmaßnahmen in Relation zur Skalierbarkeit der \textit{VC}.
\item Die Position und Privatsphäre von Nutzern und Fahrzeugen muss gewahrt werden.
\item Das Netz muss mit heterogenen Knoten aufgebaut werden können. (Die Technik in den Fahrzeugen wird aufgrund von Baujahr und Bauart auf unterschiedlichem Stand sein.)
\end{itemize}
Einige dieser Herausforderungen werden bereits durch den Einsatz von Virtualisierungstechniken gelöst.
In diesem Zusammenhang bedeutet dies, dass die Hardware der Fahrzeuge koordiniert wird, um mehrere Betriebssysteme betreiben zu können.
Währenddessen besitzt keines dieser Systeme Kenntnis von der Existenz der Anderen oder auch der Aufteilung der Ressourcen.
Auf diese Weise können mehrere Systeme parallel auf der Hardware des Fahrzeugs betrieben werden und trotzdem die Sicherheit und Privatsphäre jedes einzelnen Systems gewahrt werden.
Außerdem existieren einige Ideen zur Lösung dieser Probleme.
Eine Idee bezüglich der Position von Fahrzeugen besteht darin, die Wahrnehmung anderer Fahrzeuge per Funk und Kamera zur gegenseitigen Verifikation zu nutzen.
Zur Authentifizierung der Fahrzeuge kann auf die Public-Key-Infrastruktur \textit{(PKI)} zurückgegriffen werden, wie es bereits in \textit{VANETs} der Fall ist.
Eine weitere Idee \textit{(GeoEncrypt)} besteht darin, auf die Geolokation des Fahrzeugs zur Generierung eines Schlüssels zurückzugreifen.
Dadurch kann die Nachricht nur entschlüsselt werden, falls das Fahrzeug sich in an der korrekten Position befindet.
Um die Privatsphäre der Nutzer bei der ständigen Kommunikation und Authentifikation von Fahrzeugen zu wahren, können Pseudonyme eingesetzt werden.
Allerdings gestaltet sich die Aktualisierung der Pseudonyme aufgrund der hohen Mobilität der Fahrzeuge schwierig\cite{conti2013mobile}.

\subsection{Bedrohungen in Vehicular Clouds}
Herkömmlicherweise versuchen Angreifer in das Netz einzudringen, allerdings sind alle Nutzer der \textit{Vehicular Cloud} ein Teil des Netzes, sodass Angreifer nicht einfach ausgesperrt werden können.
Im Folgenden werden Bedrohungen beschrieben, mit denen eine \textit{VC} umgehen können müsste\cite{conti2013mobile}:
\begin{itemize}
\item Verschleierung der Identität. Der Angreifer täuscht vor, ein anderer Nutzer zu sein, um Daten oder Vorteile zu erlangen, zu denen er nicht berechtigt ist.
Ein Beispiel dafür ist der \textit{Man-in-the-Middle-Angriff}, bei dem sich der Angreifer A gegenüber als B ausgibt und B gegenüber als A ausgibt, um die Kommunikation zwischen A und B mitzuhören.
\item Manipulation. Der Angreifer manipuliert Daten, ändert Nachrichten oder verschickt eigene Nachrichten.
\item \textit{Repudiation}. Der Angreifer betreibt Manipulation mit der Identität eines anderen Nutzers, sodass die Manipulation auf den anderen Nutzer zurückzuführen ist.
\item \textit{Information Disclosure}. Der Angreifer erlangt persönliche Informationen von Nutzern, wie der Wohnort, die Herkunft oder auch medizinische Informationen.
\item \textit{Denial of Service (DoS)}. Der Nutzer versucht durch eine große Menge von gezielten Anfragen die Ressourcen der \textit{VC} zu belegen und somit die Erreichbarkeit anderer Nutzer einzuschränken.
\item Rechteausweitung. Der Angreifer nutzt eine Schwachstelle aus, um seine Zugriffsrechte innerhalb der \textit{VC} zu erhöhen.
Dadurch erlangt er Zugriff auf Daten und Ressourcen, die ihm sonst nicht zustehen würden.
Solche Zugriffsrechte sind meist Administratoren oder auch Koordinatoren des Netzes zugewiesen.
\end{itemize}

\subsection{Bedingungen für eine sichere Vehicular Cloud}
Folgende Bedingungen müssen gelten, um eine sichere Kommunikation in einer \textit{Vehicular Cloud} zu gewährleisten\cite{conti2013mobile}:
\begin{itemize}
\item Integrität. Nachrichten dürfen nicht modifiziert werden. Die Nachrichten müssen zuverlässig und korrekt sein.
\item Vertraulichkeit. Unautorisierte Nutzer dürfen nicht auf sensible Nachrichten zugreifen.
\item Verfügbarkeit. Nachrichten müssen verfügbar sein, wenn sie benötigt werden.
\item Authentifizierung. Nachrichten müssen auf eine rechtmäßige Weise von autorisierten Knoten verschickt werden.
Es werden sowohl der Absender als auch der Inhalt einer Nachricht authentifiziert.
\item \textit{Anti-DoS}. Die \textit{VC} muss fähig sein einen \textit{DoS-Angriff} abzuwehren.
\item Echtzeitfähigekit. Manche Anwendungen und Ereignisse, wie beispielsweise bei einem Unfall, sind zeitkritisch.
Daher muss die \textit{VC} fähig sein, zeitkritische Nachrichten übertragen zu können.
\item Privatsphäre. Die Privatsphäre der Nutzer muss gewahrt werden.
\item Sybil-Attacken-frei. Nachrichten von nicht authentifizierten Nutzern dürfen nicht übertragen werden.
\end{itemize}
% ----------------------------------------------------------------------------------------------------------------------
\section{Virtualisierung}
\label{sec:aggregation}

In Kapitel \ref{sec:securityprivacy} wurde bereits von Virutalisierung gesprochen, um einige Sicherheitsprobleme zu lösen.
Im Folgenden sollen zwei Ansätze für die Verteilung der Ressourcen in \textit{Vehicular Clouds} erläutert werden.
Dabei wird auf die Terminologie von \textit{Dedicated Short-Range Communications (DSRC)} zurückgegriffen.
Die Rechenleistung eines Fahrzeugs wird mit dem Begriff \textit{On-Board-Unit (OBU)} bezeichnet.
An der Straße befindliche Infrastruktur wird mit dem Begriff \textit{Roadside-Unit (RSU)} bezeichnet.
Zusätzlich wird auf die Terminologie von \textit{Cloud Computing} zurückgegriffen, und zwar die drei Begriffe Knoten \textit{(Node)}, \textit{Cluster} und \textit{Cloud Controller}.
Wie aus Kapitel \ref{sec:cc} bekannt, können damit \textit{IaaS-}, \textit{PaaS-} und \textit{SaaS-}Dienste angeboten werden.
Die Mobilität des jeweiligen Fahrzeugs bestimmt dabei die verwendete Netzwerkinfrastruktur.
Sich bewegende Fahrzeuge greifen zwangsläufig auf kabellose Kommunikation wie \textit{WiFi}, 5G oder auch \textit{DSRC} zurück, während stehende Fahrzeuge die Möglichkeit besitzen, mit einer Ethernet Verbindung ausgestattet zu werden.
Die Architektur von \textit{CC} ist darauf ausgelegt, eine Mindestdatenrate von 1Gbits zu besitzen.
Zusätzlich läuft der \textit{Node Controller} auf einem System mit Mehrkernprozessoren, Arbeitsspeicher mit hoher Dichte und teilweise sogar mehreren Prozessoren.
Daher können die voraussichtlich geringen Hardware Spezifikationen der Fahrzeuge ein Problem darstellen.
Somit ist es eventuell notwendig, mehrere \textit{OBUs} für den Betrieb einer Virtuellen Maschine (VM) zu bündeln\cite{conti2013mobile}.

\subsection{Herkömmlicher Ansatz}
Aktuelle Implemtierungen von Virtualisierung bauen, wie in Abbildung \ref{fig:fullVirtual} gezeigt, auf der Idee auf, dass eine sogenannte \textit{Hypervisor}-Software die VMs kontrolliert und ihnen die Ressourcen zuweist.
Abbildung \ref{fig:obuVirtual} zeigt die nowtendigen Anpassungen für eine Verwendung in \textit{VCs}.
Mehrere \textit{OBUs} werden gebündelt und der Datenfluss durch einen sogenannten \textit{Resource Scheduler (RS)} kontrolliert.
Somit lassen sich die verfügbaren Ressourcen einer VM durch die Hinzunahme beziehungsweise Entfernung von \textit{OBUs} anpassen.
Die Basisblöcke von CPU-Instruktionen würden durch den RS an die \textit{OBUs} verteilt werden.

\begin{figure}[h]
\centering
\includegraphics[width=0.5\textwidth]{images/fullVirtual}
\caption{Herkömmliche Virtualisierung\cite{conti2013mobile}}
\label{fig:fullVirtual}
\end{figure}
\begin{figure}[h]
\centering
\includegraphics[width=0.5\textwidth]{images/obuVirtual}
\caption{Angepasste Virtualisierung für \textit{Vehicular Clouds}\cite{conti2013mobile}}
\label{fig:obuVirtual}
\end{figure}

Die durchschnittliche Datenrate von DDR4-Arbeitsspeicher liegt zwischen 2 und 3Gbits und somit eine Mindestanforderung für die Funktion einer \textit{VC}.
Eine Datenrate von 40Gbits wäre notwendig, um eine 2Gbits Übertragung von Basisblöcken zu ermöglichen\cite{conti2013mobile}.
Da die derzeit unterstützte Datenrate von \textit{DSRC} bei 27Gbits liegt, ist eine Umsetzung dieses Ansatzes momentan nicht denkbar\cite{advTrans2017}.

\subsection{Lastverteilung}
Ein weiterer Ansatz bietet sich besonders für Anwendungsfälle wie das in Kapitel \ref{sec:flughafen} vorgestellte Beispiel an.
In Abbildung \ref{fig:loadBalance} wird der Aufbau des Lastverteilungs-Ansatzes dargestellt.
Der \textit{Cluster Controller} befindet sich in diesem Fall auf den \textit{RSUs} und kommunziert mit den \textit{Node Controllern}, welche auf den \textit{OBUs} laufen.
Bei der Ankunft neuer Fahrzeuge werden die \textit{OBUs} von einer \textit{Load Balancer}-Komponente des \textit{Cluster Controllers} registriert und einer VM zugewiesen.
Der \textit{Load Balancer} speichert Informationen über die VMs wie die zugrundeliegenden \textit{OBUs}, IP und die angebotenen Dienste ab und weist eingehende Anfragen möglichst effizient den VMs zu.
Dazu können Zufallsalgorithmen oder beispielsweise \textit{Round Robin} angewandt werden\cite{conti2013mobile}.

\begin{figure}[h]
\centering
\includegraphics[width=0.5\textwidth]{images/loadBalance}
\caption{Load Balancing und Bildung von Virtuellen Maschinen\cite{conti2013mobile}}
\label{fig:loadBalance}
\end{figure}

\subsubsection{Registrierung von Knoten}
Bei der Ankunft eines neuen Fahrzeugs nimmt dieses die Kommunikation mit der nächsten \textit{RSU} auf.
Daraufhin wird der \textit{Cluster Controller} den Knoten einer VM zuordnen und ein Image eines Betriebssystems an den \textit{Node Controller} senden.
Dieses Image ist darauf angepasst, dass es den für die VM benötigten Dienst bereitstellen kann.
Somit hängt die Bereitstellungsdauer des Dienstes von dem gewählten Kommunikationsweg und der Größe des Betriebssystems ab.
Dieser Ansatz ermöglicht die Aufteilung eines Dienstes in kleinere Images, welche gemeinsam eine einzelne große VM bilden\cite{conti2013mobile}.

\subsubsection{Abmeldung von Knoten}
Wenn ein Fahrzeug den \textit{Cluster} verlässt, muss der \textit{Cluster Controller} sicherstellen, dass alle Anfragen weiterhin verarbeitet werden.
Zusätzlich müssen die Antwortzeiten der Dienste, welche von der betroffenen VM bereitgestellt werden, in einem akzeptablen Maß bleiben.
Dieser Vorgang beginnt mit dem Versand einer Nachricht durch das Fahrzeug an den \textit{Cluster Controller}, dass dieses Fahrzeug sich abmelden möchte.
Daraufhin wird gewartet, ob sich ein neues Fahrzeug anmelden möchte.
Sollte in dem zuvor festgelegten Zeitraum sich kein neues Fahrzeug anmelden, werden benachbarte \textit{Cluster Controller} gefragt, ob sie einen Ersatz bereitstellen können.
Dabei muss sich der Ersatz allerdings in Kommunikationsreichweite des anfragenden \textit{Cluster Controllers} befinden\cite{conti2013mobile}.

% ----------------------------------------------------------------------------------------------------------------------
\section{Fazit und Ausblick}
\label{sec:conclusion}

Mithilfe des Konzepts der \textit{Vehicular Clouds} wurde ein Einblick in die zukünftigen Entwicklungen von Fahrzeug-Ad-hoc-Netzen \textit{(VANETs)} gegeben.
Die \textit{Vehicular Cloud} ist eine Weiterentwicklung der \textit{VANETs}, welche ungenutzte Ressourcen von Fahrzeugen wie Rechenleistung, Internetverbindung und Speicher nutzt, um sie mit anderen Fahrzeugen zu teilen oder an Dritte zu vermieten.
Es wurde gezeigt, dass auf diese Weise Verkehrsereignisse abgewendet werden können, indem die Ressourcen der beteiligten Fahrzeuge verwendet werden.
Außerdem wurde eine Abwandlung eines Rechenzentrums vorgestellt, welche mittels der Fahrzeugressourcen Dienste an Dritte bereitstellen kann.
Dabei entstehende Sicherheitsprobleme und Bedenken hinsichtlich der Privatsphäre wurden genannt und teilweise Lösungsansätze vorgestellt.
Schließlich wurden die Struktur von \textit{Vehicular Clouds} und Ansätze für die Zuweisung von Ressourcen erläutert.

Es ist davon auszugehen, dass \textit{VANETs} unseren Straßenverkehr in der nahen Zukunft sicherer gestalten werden und damit einhergehend die Ressourcen der Fahrzeuge, wie die Rechenleistung oder auch der vorhandene Speicher, zunehmen werden.
Wenn diese Ressourcen nicht gänzlich ausgeschöpft werden, bietet sich dazu eine Verwendung für \textit{Cloud Computing} an.
Da für die eingebetteten Systeme der Fahrzeuge allerdings Limitierungen wie die Mobilität oder auch die Geschwindigkeit der Verbindung herrschen, müssen die klassischen Konzepte der Virtualisierung angepasst werden.

% ----------------------------------------------------------------------------------------------------------------------
% \subsubsection*{Acknowledgments}
% \ldots

\newpage

% In the bibliography, use \texttt{\textbackslash textsuperscript} for \qq{st}, \qq{nd}, \ldots:
% E.g., \qq{The 2\textsuperscript{nd} conference on examples}.
% When you use \href{https://www.jabref.org}{JabRef}, you can use the clean up command to achieve that.
% See \url{https://help.jabref.org/en/CleanupEntries} for an overview of the cleanup functionality.

\renewcommand{\bibsection}{\section*{References}} % requried for natbib to have "References" printed and as section*, not chapter*
% Use natbib compatbile splncsnat style.
% It does provide all features of splncs03, but is developed in a clean way.
% Source: http://phaseportrait.blogspot.de/2011/02/natbib-compatible-bibtex-style-bst-file.html
\bibliographystyle{splncsnat}
\begingroup
  \ifluatex
    %try to activate if bibliography looks ugly
    %\sloppy
  \else
    \microtypecontext{expansion=sloppy}
  \fi
  \small % ensure correct font size for the bibliography
  \bibliography{paper}
\endgroup

% Enfore empty line after bibliography
\ \\
%
Alle Links wurden zuletzt am 26. Januar 2021 aufgerufen.
\end{document}
